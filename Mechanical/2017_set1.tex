\section*{Q1}
Question 1 The product of eigenvalues of the matrix P is [latex]P=\begin{bmatrix} 2 & 0 & 1\\ 4 & -3 & 3\\ 0 & 2 & -1 \end{bmatrix}[/latex]
\textbf{Options:}
\textbf{optionQ1_A:} -6
\textbf{optionQ1_B:} 2
\textbf{optionQ1_C:} 6
\textbf{optionQ1_D:} -2
\textbf{solutionQ1:} Product of Eigen value = determinant value [latex] \begin{array}{l} =2(3-6)+1(8-0) \\ =2(-3)+8=-6+8=2 \end{array}[/latex]

\section*{Q2}
Question 2 The value of  [latex]lim \_{x \to 0}\: \frac{x^{3}-\sin(x)}{x}[/latex]  is
\textbf{Options:}
\textbf{optionQ2_A:} 0
\textbf{optionQ2_B:} 3
\textbf{optionQ2_C:} 1
\textbf{optionQ2_D:} -1
\textbf{solutionQ2:} [latex]\begin{aligned} \lim \_{x \rightarrow 0} \frac{x^{3}-\sin x}{x} &=\left(\frac{0}{0} \mathrm{form}\right) \\ \lim \_{x \rightarrow 0} \frac{3 x^{2}-\cos x}{1} &=0-\cos 0 \\ &=0-1=-1 \end{aligned}[/latex]

\section*{Q3}
Question 3 consider the following partial differential equation for u(x,y) with the constant c[latex]\gt 1[/latex]: [latex]\frac{\partial y}{\partial x}+c \: \frac{\partial u}{\partial x} =0[/latex] solution of this equations is
\textbf{Options:}
\textbf{optionQ3_A:} [latex]u(x,y)=f(x+cy)[/latex]
\textbf{optionQ3_B:} [latex]u(x,y)=f(x-cy)[/latex]
\textbf{optionQ3_C:} [latex]u(x,y)=f(cx+y)[/latex]
\textbf{optionQ3_D:} [latex]u(x,y)=f(cx-y)[/latex]
\textbf{solutionQ3:} [latex]\small \begin{aligned} u &=f(x-c y) \\ \frac{\partial u}{\partial x} &=f^{\prime}(x-c y)(1) \\ \frac{\partial u}{\partial y} &=f^{\prime}(x-c y)(-c) \\ &=-c \cdot f^{\prime}(x-c y)\\ &=-c \cdot \frac{\partial u}{\partial x} \\ \therefore  \quad \frac{\partial u}{\partial y} & +c \frac{\partial u}{\partial x} =0 \end{aligned}[/latex]

\section*{Q4}
Question 4 The differential equation [latex]\frac{\mathrm{d^{2}}y}{\mathrm{d} x^{2}}+16y=0[/latex] for y(x) with the two boundary conditions [latex]\left | \frac{\mathrm{d} y}{\mathrm{d} x} \right |\_{x=0}=1[/latex] and  [latex]\left | \frac{\mathrm{d} y}{\mathrm{d} x} \right |\_{x=\frac{\pi}{2}}=-1[/latex] has
\textbf{optionQ4_A:} no solution
\textbf{optionQ4_B:} exactly two solutions
\textbf{optionQ4_C:} exactly one solutions
\textbf{optionQ4_D:} infinitely many solutions
\textbf{solutionQ4:} [latex] \begin{aligned} \left(d^{2}+16\right) y &=0 \\ A E \text { is } m^{2}+16 &=0 \\ m &=\pm 4 i \end{aligned}[/latex] Solution is [latex] y=c\_{1} \cos 4 x+c\_{2} \sin 4 x[/latex] [latex] \begin{aligned} y' &=-4 c\_{1} \sin 4 x+4 c\_{2} \cos 4 x \\ y'(0) &=1 \\ 1 &=4 c\_{2} \\ c\_{2} &=1 / 4 \\ y'(\pi / 2) &=-1 \\ -1 &=-4 c\_{1} \sin 2 \pi+4 c\_{2} \cos 2 \pi \\ -1 &=0+4 c\_{2} \\ c\_{2} &=-1 / 4 \end{aligned}[/latex] Therefore the given differential equation has no solution.

\section*{Q5}
Question 5 A six-face fair dice is rolled a large number of times. The mean value of the outcomes is \_\_\_\_\_\_\_\_
\textbf{Options:}
\textbf{solutionQ5:} [latex]\begin{aligned} \text{mean} &=E(x)=\Sigma x \cdot P(x) \\ &=1(1 / 6)+2(1 / 6)+3(1 / 6)+4 \\ &(1 / 6)+5(1 / 6)+6(1 / 6) \\ &=\frac{1}{6}(1+2+3+4+5+6) \\ &=\frac{21}{6}=3.5 \end{aligned}[/latex]

\section*{Q6}
Question 4 The differential equation [latex]\frac{\mathrm{d^{2}}y}{\mathrm{d} x^{2}}+16y=0[/latex] for y(x) with the two boundary conditions [latex]\left | \frac{\mathrm{d} y}{\mathrm{d} x} \right |\_{x=0}=1[/latex] and  [latex]\left | \frac{\mathrm{d} y}{\mathrm{d} x} \right |\_{x=\frac{\pi}{2}}=-1[/latex] has
\textbf{optionQ4_A:} no solution
\textbf{optionQ4_B:} exactly two solutions
\textbf{optionQ4_C:} exactly one solutions
\textbf{optionQ4_D:} infinitely many solutions
\textbf{solutionQ4:} [latex] \begin{aligned} \left(d^{2}+16\right) y &=0 \\ A E \text { is } m^{2}+16 &=0 \\ m &=\pm 4 i \end{aligned}[/latex] Solution is [latex] y=c\_{1} \cos 4 x+c\_{2} \sin 4 x[/latex] [latex] \begin{aligned} y' &=-4 c\_{1} \sin 4 x+4 c\_{2} \cos 4 x \\ y'(0) &=1 \\ 1 &=4 c\_{2} \\ c\_{2} &=1 / 4 \\ y'(\pi / 2) &=-1 \\ -1 &=-4 c\_{1} \sin 2 \pi+4 c\_{2} \cos 2 \pi \\ -1 &=0+4 c\_{2} \\ c\_{2} &=-1 / 4 \end{aligned}[/latex] Therefore the given differential equation has no solution.


